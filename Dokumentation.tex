\documentclass[12pt,a4paper,bibliography=totocnumbered,listof=totocnumbered]{scrartcl}
\usepackage[ngerman]{babel}
\usepackage[utf8]{inputenc}
\usepackage{amsmath}
\usepackage{amsfonts}
\usepackage{amssymb}
\usepackage{graphicx}
\usepackage{fancyhdr}
\usepackage{tabularx}
\usepackage{geometry}
\usepackage{setspace}
\usepackage[right]{eurosym}
\usepackage[printonlyused]{acronym}
\usepackage{subfig}
\usepackage{floatflt}
\usepackage[usenames,dvipsnames]{color}
\usepackage{colortbl}
\usepackage{paralist}
\usepackage{array}
\usepackage{titlesec}
\usepackage{parskip}
\usepackage[right]{eurosym}
%\usepackage{wrapfig}
%\usepackage[backend=biber, citestyle=authoryear]{biblatex}

\usepackage[subfigure,titles]{tocloft}
\usepackage[pdfpagelabels=true]{hyperref}

\usepackage{listings}
\lstset{basicstyle=\footnotesize, captionpos=b, breaklines=true, showstringspaces=false, tabsize=2, frame=lines, numbers=left, numberstyle=\tiny, xleftmargin=2em, framexleftmargin=2em}
\makeatletter
\def\l@lstlisting#1#2{\@dottedtocline{1}{0em}{1em}{\hspace{1,5em} Lst. #1}{#2}}
\makeatother

\geometry{a4paper, top=27mm, left=30mm, right=20mm, bottom=35mm, headsep=10mm, footskip=12mm}

\hypersetup{unicode=false, pdftoolbar=true, pdfmenubar=true, pdffitwindow=false, pdfstartview={FitH},
	pdftitle={Projektkurs Dokumentation},
	pdfauthor={Jakob Fleischer},
	pdfsubject={Dokumentation},
	pdfcreator={\LaTeX\ with package \flqq hyperref\frqq},
	pdfproducer={pdfTeX \the\pdftexversion.\pdftexrevision},
	pdfkeywords={Dokumentation},
	pdfnewwindow=true,
	colorlinks=true,linkcolor=black,citecolor=black,filecolor=magenta,urlcolor=black}
\pdfinfo{/CreationDate \today}

%\bibliography{bibo.bib}
%\addbibresource{bibo.bib}

\begin{document}

\titlespacing{\section}{0pt}{12pt plus 4pt minus 2pt}{-6pt plus 2pt minus 2pt}

% Kopf- und Fusszeile
\renewcommand{\sectionmark}[1]{\markright{#1}}
\renewcommand{\leftmark}{\rightmark}
\pagestyle{fancy}
\lhead{}
\chead{}
\rhead{\thesection\space\contentsname}
\lfoot{Dokumentation unseres Projekts\newline im Projektkurs Mathe-Physik-Informatik}
\cfoot{}
\rfoot{\ \linebreak Seite \thepage}
\renewcommand{\headrulewidth}{0.4pt}
\renewcommand{\footrulewidth}{0.4pt}

% Vorspann
\renewcommand{\thesection}{\Roman{section}}
\renewcommand{\theHsection}{\Roman{section}}
\pagenumbering{Roman}

% ----------------------------------------------------------------------------------------------------------
% Titelseite
% ----------------------------------------------------------------------------------------------------------
\thispagestyle{empty}
\begin{center}
	\includegraphics[scale=2]{Bilder/aeg-logo.png}\\
	
	\Large
	\textbf{Albert-Einstein-Gymnasium}\\
	%\textbf{Ingenieurwissenschaften und Informatik}\\
	\vspace*{2cm}
	\vspace*{2cm}
	\Huge
	\textbf{Dokumentation}\\
	%\textbf{des Prjektkurses Mathe-Physik-Informatik}
	\vspace*{0.5cm}
	\large
	zu meinem Projekt im Projektkurs\\
	\vspace*{1cm}
	\Huge
	\textbf{Mathe-Physik-Informatik}\\
	\vspace*{2cm}
	
	\vfill
	\normalsize
	\newcolumntype{x}[1]{>{\raggedleft\arraybackslash\hspace{0pt}}p{#1}}
	\begin{tabular}{x{6cm}p{7.5cm}}
		\rule{0mm}{5ex}\textbf{Autor:} & Jakob Fleischer\newline jakotho@gmx.de \\ 
		\rule{0mm}{5ex}\textbf{Prüfer:} & Herr Thomas Bachran \\ 
		\rule{0mm}{5ex}\textbf{Abgabedatum:} & 21.06.2017 \\ 
	\end{tabular} 
\end{center}
\pagebreak

% ----------------------------------------------------------------------------------------------------------
% Abstract
% ----------------------------------------------------------------------------------------------------------
\setcounter{page}{1}
\onehalfspacing
\titlespacing{\section}{0pt}{12pt plus 4pt minus 2pt}{2pt plus 2pt minus 2pt}
\rhead{KURZFASSUNG}
\section{Umschreibung des Projekts}
...

\vspace{-1,2em}
\titlespacing{\section}{0pt}{12pt plus 4pt minus 2pt}{-6pt plus 2pt minus 2pt}

\pagebreak

% ----------------------------------------------------------------------------------------------------------
% Verzeichnisse
% ----------------------------------------------------------------------------------------------------------
% TODO Typ vor Nummer
\renewcommand{\cfttabpresnum}{Tab. }
\renewcommand{\cftfigpresnum}{Abb. }
\settowidth{\cfttabnumwidth}{Abb. 10\quad}
\settowidth{\cftfignumwidth}{Abb. 10\quad}

\titlespacing{\section}{0pt}{12pt plus 4pt minus 2pt}{2pt plus 2pt minus 2pt}
\singlespacing
\rhead{INHALTSVERZEICHNIS}
\renewcommand{\contentsname}{II Inhaltsverzeichnis}
\phantomsection
\addcontentsline{toc}{section}{\texorpdfstring{II \hspace{0.35em}Inhaltsverzeichnis}{Inhaltsverzeichnis}}
\addtocounter{section}{1}
\tableofcontents
\pagebreak


\rhead{VERZEICHNISSE}
\listoffigures
\pagebreak
\listoftables
\pagebreak
\renewcommand{\lstlistlistingname}{Listing-Verzeichnis}
{\labelsep2cm\lstlistoflistings}
\pagebreak

% ----------------------------------------------------------------------------------------------------------
% Abkürzungen
% ----------------------------------------------------------------------------------------------------------
%\section{Abkürzungsverzeichnis}
%\begin{acronym}[OSGi] % längste Abkürzung steht in eckigen Klammern
%	\setlength{\itemsep}{-\parsep} % geringerer Zeilenabstand
%	\acro{OSGi}{Open Service Gateway initiative}
%\end{acronym}
%\newpage

% ----------------------------------------------------------------------------------------------------------
% Inhalt
% ----------------------------------------------------------------------------------------------------------
% Abstände Überschrift
\titlespacing{\section}{0pt}{12pt plus 4pt minus 2pt}{-6pt plus 2pt minus 2pt}
\titlespacing{\subsection}{0pt}{12pt plus 4pt minus 2pt}{-6pt plus 2pt minus 2pt}
\titlespacing{\subsubsection}{0pt}{12pt plus 4pt minus 2pt}{-6pt plus 2pt minus 2pt}

% Kopfzeile
\renewcommand{\sectionmark}[1]{\markright{#1}}
\renewcommand{\subsectionmark}[1]{}
\renewcommand{\subsubsectionmark}[1]{}
\lhead{Kapitel \thesection}
\rhead{\rightmark}

\onehalfspacing
\renewcommand{\thesection}{\arabic{section}}
\renewcommand{\theHsection}{\arabic{section}}
\setcounter{section}{0}
\pagenumbering{arabic}
\setcounter{page}{1}

% ----------------------------------------------------------------------------------------------------------
% Einleitung
% ----------------------------------------------------------------------------------------------------------
\section{Kapitel: Herangehensweise}
Bei der Planung und der Herangehensweise an unser Projekt sind wir vermehrt auf Probleme gestoßen, welche ich
in diesem Kapitel darstellen möchte.

...
\subsection{ursprüngliche Vorstellung}
Unsere Unerfahrenheit in der Projektarbeit hat uns zu einem überstürzten und unüberlegten Projekteinstieg getrieben, bei dem wir uns in Themen eingearbeitet haben, ohne uns einen genauen Plan zu erstellen.

...
\subsection{Konsequenz}
Zu spät haben wir gemerkt, dass unsere bisherige Arbeit weder zielführend, noch in irgendeiner Weise sinnvoll war, weshalb wir uns zusammengesetzt und intensiv beraten haben.
....
\subsection{verspätete Planung}
...
\subsection{Zusammenarbeit und Kommunikation}
....

\section{Kapitel: Themen}
Im zweiten Kapitel dieser Dokumentation möchte ich auf die Themen eingehen, in welche ich mich im Laufe des Projektkurses hineingearbeitet habe.
Diese sind nicht chronologisch, sondern nach Relevanz für unser Projekt geordnet, da wir manche Vorgehensweisen wieder verworfen haben.
Somit werde ich zunächst auf die verworfenen Themen und dann auf die projektrelevanten Themen eingehen.

\subsection{Java-Servlet}
Im Folgenden werde ich erläutern, wie man ein Servlet in Eclipse über einen Apache Tomcat Server erstellen kann.

\subsubsection{Einrichtung in Eclipse}
Vorausgesetzt für eine erfolgreiche Einrichtung ist die bereits erfolgte Installation von Eclipse Neo in der Java EE Version und Apache Tomcat v 9.0. 
Nachdem  diese erfolgt ist, ist es nun möglich in Eclipse einen neuen Server einzurichten, wobei man unter dem Ordner Apache die installierte Tomcat Version vorfindet, welche hier  als Servertyp verwendet wird.
Nachdem dies geschehen ist, richten sie ein Dynamic Web Project ein und innerhalb desselben erstellen sie im Java Resources/src  Ordner ein neues Servlet, indem sie nun programmieren können.

Wenn nun aber die Fehlermeldung  „cannot resolve“ im Bezug auf die vorgegebenen import-Zeilen auftaucht müssen sie außerdem noch die servlet-api.jar downloaden und diese dann in den Eigenschaften ihres Dynamic Web Projects unter Java Build Path in der Kategorie Libraries mit der Funktion „Add External JARs“ hinzufügen. 

...

\subsubsection{Vorteile}
...
\subsubsection{Nachteile}
...
%\vspace{1em}
%\begin{minipage}{\linewidth}
%	\centering
%	\includegraphics[width=0.7\linewidth]{Bilder/layering-osgi.png}
%	\captionof{figure}[OSGi Architektur]{OSGi Architektur\footnotemark }
%	\label{fig:osgi}
%\end{minipage}
%\footnotetext{Quelle: \url{http://www.osgi.org/Technology/WhatIsOSGi}}

\subsection{JSON}
Ein weiteres Thema, in welches ich mich eingearbeitet habe war die Erstellung und das Auslesen von Dateien
des Formates JSON mit Eclipse.

...

\subsubsection{Erstellen}
...
\subsubsection{Auslesen}
...
\subsubsection{Vorteile}
...
\subsubsection{Nachteile}
...
%\vspace{1em}
%\begin{table}[!h]
%	\centering
%	\begin{tabular}{|l|l|l|}
%		\hline
%		\textbf{Name} & \textbf{Name} & \textbf{Name}\\
%		\hline
%		1 & 2 & 3\\
%		\hline
%		4 & 5 & 6\\
%		\hline
%		7 & 8 & 9\\
%		\hline
%	\end{tabular}
%	\caption{Beispieltabelle}
%	\label{tab:beispiel}
%\end{table}

\pagebreak
\subsection{MYSQL}
Nachdem obige Thematiken angesichts unseres erneuerten Projektplans irrelevant geworden waren, habe ich mich 
mit MYSQL beschäftigt.

...
\subsubsection{Vorteile}
%\begin{compactitem}
%	\item Nur
%	\item ein
%	\item Beispiel.
%\end{compactitem}

\subsubsection{Datenbank-Entwurf}
Im Folgenden möchte ich auf das Entwerfen einer MYSQL-Datenbank eingehen.
...

% \ref{lst:arduino}.

%\vspace{1em}
%\begin{lstlisting}[caption=Arduino Beispielprogramm, label=lst:arduino]
%int ledPin = 13;
%void setup() {
 %   pinMode(ledPin, OUTPUT);
%}
%void loop() {
 %   digitalWrite(ledPin, HIGH);
  %  delay(500);
   % digitalWrite(ledPin, LOW);
    %delay(500);
%}
%\end{lstlisting}

\subsubsection{Verwaltung einer Datenbank mit PHP}
Um eine Datenbank in unser Projekt zweckmäßig einzurichten, musste ich mich in eine MYSQL-Einbindung mittels PHP einlesen, welche ich nun darstellen möchte.

...

Die Quellen befinden sich in der Datei \textit{bibo.bib}. Ein Buch- und eine Online-Quelle sind beispielhaft eingefügt. [Vgl. \cite{buch}, \cite{online}]

%Abkürzungen lassen sich natürlich auch nutzen (\ac{OSGi}). Weiter oben im Latex-Code findet sich das %Verzeichnis.
\pagebreak

% ----------------------------------------------------------------------------------------------------------
% Kapitel
% ----------------------------------------------------------------------------------------------------------
\section{Projekt}
In diesem Kapitel stelle ich nun unser Projekt dar, indem ich zunächst das Programm im Gesamten beschreibe, daraufhin kurz auf die Arbeit von Lars und Robin eingehe, woraufhin ich dann meinen Anteil gründlcih darlege.


...

\subsection{File-System-Management-System}
Darstellung Ergebnis....

\subsection{GUI - Lars}
Darstellung GUI....

\subsection{Engine I: Interface und File-System - Robin}
Darstellung Robins Part....

\subsection{Engine II: Interface und Datenbank - Ich}
Darstellung mein Ergebnis...
\pagebreak
% ----------------------------------------------------------------------------------------------------------
% Kapitel
% ----------------------------------------------------------------------------------------------------------
\section{Fazit}
...
\pagebreak
% ----------------------------------------------------------------------------------------------------------
% Literatur
% ----------------------------------------------------------------------------------------------------------

%\addcontentsline{tocnumbered}{section}{Literaturverzeichnis}

\begin{thebibliography}{9}

%\bibitem[Euklid]{geo} \emph{Die Elemente},
%Euklid 300 v. Chr.
%\bibitem[Pythagoras]{dreieck} \emph{Satz des Pythagoras},
%Pythagoras 520 v. Chr.
\end{thebibliography}
%\section{Quellenverzeichnis}
%\printbibliography
%\renewcommand\refname{Quellenverzeichnis}
%\bibliographystyle{alpha}
%\bibliography{bibo}
\pagebreak


% ----------------------------------------------------------------------------------------------------------
% Anhang
% ----------------------------------------------------------------------------------------------------------
\pagenumbering{Roman}
\setcounter{page}{1}

\lhead{Anhang \thesection}
\begin{appendix}

\section*{Anhang}
\phantomsection
\addcontentsline{toc}{section}{Anhang}
\addtocontents{toc}{\vspace{-0.5em}}


%\section{GUI}
%Ein toller Anhang.

%\subsection*{Screenshot}
%\label{app:screenshot}
%Unterkategorie, die nicht im Inhaltsverzeichnis %auftaucht.

\end{appendix}


\newpage
\thispagestyle{empty}
\begin{center}
	\vspace*{5em}
	\huge\textbf{Erklärung}\\
\end{center}
\vspace{2em}
Hiermit versichere ich, dass ich meine Dokumentation selbständig verfasst und keine anderen als die angegebenen Quellen und Hilfsmittel benutzt habe.

\vspace{4em}
\begin{minipage}{\linewidth}
	\begin{tabular}{p{15em}p{15em}}
		Datum: &  .......................................................\\
		& \centering (Unterschrift)\\
	\end{tabular}
\end{minipage}

\end{document}
