\documentclass[12pt,a4paper,bibliography=totocnumbered,listof=totocnumbered]{scrartcl}
\usepackage[ngerman]{babel}
\usepackage[utf8]{inputenc}
\usepackage{amsmath}
\usepackage{amsfonts}
\usepackage{amssymb}
\usepackage{graphicx}
\usepackage{fancyhdr}
\usepackage{tabularx}
\usepackage{geometry}
\usepackage{setspace}
\usepackage[right]{eurosym}
\usepackage[printonlyused]{acronym}
\usepackage{subfig}
\usepackage{floatflt}
\usepackage[usenames,dvipsnames]{color}
\usepackage{colortbl}
\usepackage{paralist}
\usepackage{array}
\usepackage{titlesec}
\usepackage{parskip}
\usepackage[right]{eurosym}
%\usepackage{url}
%\usepackage{wrapfig}
%\usepackage[backend=biber, citestyle=authoryear]{biblatex}

\usepackage[subfigure,titles]{tocloft}
\usepackage[pdfpagelabels=true]{hyperref}

\usepackage{listings}
\lstset{basicstyle=\footnotesize, captionpos=b, breaklines=true, showstringspaces=false, tabsize=2, frame=lines, numbers=left, numberstyle=\tiny, xleftmargin=2em, framexleftmargin=2em}
\makeatletter
\def\l@lstlisting#1#2{\@dottedtocline{1}{0em}{1em}{\hspace{1,5em} Lst. #1}{#2}}
\makeatother

\geometry{a4paper, top=27mm, left=30mm, right=20mm, bottom=35mm, headsep=10mm, footskip=12mm}

\hypersetup{unicode=false, pdftoolbar=true, pdfmenubar=true, pdffitwindow=false, pdfstartview={FitH},
	pdftitle={Projektkurs Dokumentation},
	pdfauthor={Jakob Fleischer, Robin Schlaak, Lars Schmalbach},
	pdfsubject={Dokumentation},
	pdfcreator={\LaTeX\ with package \flqq hyperref\frqq},
	pdfproducer={pdfTeX \the\pdftexversion.\pdftexrevision},
	pdfkeywords={Dokumentation},
	pdfnewwindow=true,
	colorlinks=true,linkcolor=black,citecolor=black,filecolor=magenta,urlcolor=black}
\pdfinfo{/CreationDate \today}

%\bibliography{bibo.bib}
%\addbibresource{bibo.bib}

\begin{document}

\titlespacing{\section}{0pt}{12pt plus 4pt minus 2pt}{-6pt plus 2pt minus 2pt}

% Kopf- und Fusszeile
\renewcommand{\sectionmark}[1]{\markright{#1}}
\renewcommand{\leftmark}{\rightmark}
\pagestyle{fancy}
\lhead{}
\chead{}
\rhead{\thesection\space\contentsname}
\lfoot{Dokumentation unseres Projekts\newline im Projektkurs Mathe-Physik-Informatik}
\cfoot{}
\rfoot{\ \linebreak Seite \thepage}
\renewcommand{\headrulewidth}{0.4pt}
\renewcommand{\footrulewidth}{0.4pt}

% Vorspann
\renewcommand{\thesection}{\Roman{section}}
\renewcommand{\theHsection}{\Roman{section}}
\pagenumbering{Roman}

% ----------------------------------------------------------------------------------------------------------
% Titelseite
% ----------------------------------------------------------------------------------------------------------
\thispagestyle{empty}
\begin{center}
	\includegraphics[scale=2]{Bilder/aeg-logo.png}\\
	
	\Large
	\textbf{Albert-Einstein-Gymnasium}\\
	\vspace*{2cm}
	\vspace*{2cm}
	\Huge
	\textbf{Dokumentation}\\
	\vspace*{0.5cm}
	\large
	zu unserem Projekt im Projektkurs\\
	\vspace*{1cm}
	\Huge
	\textbf{Mathe-Physik-Informatik}\\
	\vspace*{2cm}
	
	\vfill
	\normalsize
	\newcolumntype{x}[1]{>{\raggedleft\arraybackslash\hspace{0pt}}p{#1}}
	\begin{tabular}{x{6cm}p{7.5cm}}
		\rule{0mm}{5ex}\textbf{Autoren:} & Jakob Fleischer\newline Robin Schlaak \newline
		Lars Schmalbach \\ 
		\rule{0mm}{5ex}\textbf{Prüfer:} & Herr Thomas Bachran \\ 
		\rule{0mm}{5ex}\textbf{Abgabedatum:} & 21.06.2017 \\ 
	\end{tabular} 
\end{center}
\pagebreak

% ----------------------------------------------------------------------------------------------------------
% Abstract
% ----------------------------------------------------------------------------------------------------------
\setcounter{page}{1}
\onehalfspacing
\titlespacing{\section}{0pt}{12pt plus 4pt minus 2pt}{2pt plus 2pt minus 2pt}
\rhead{KURZFASSUNG}
\section{Umschreibung des Projekts}
\emph{Jakob}
...

\vspace{-1,2em}
\titlespacing{\section}{0pt}{12pt plus 4pt minus 2pt}{-6pt plus 2pt minus 2pt}

\pagebreak

% ----------------------------------------------------------------------------------------------------------
% Verzeichnisse
% ----------------------------------------------------------------------------------------------------------
% TODO Typ vor Nummer
\renewcommand{\cfttabpresnum}{Tab. }
\renewcommand{\cftfigpresnum}{Abb. }
\settowidth{\cfttabnumwidth}{Abb. 10\quad}
\settowidth{\cftfignumwidth}{Abb. 10\quad}

\titlespacing{\section}{0pt}{12pt plus 4pt minus 2pt}{2pt plus 2pt minus 2pt}
\singlespacing
\rhead{INHALTSVERZEICHNIS}
\renewcommand{\contentsname}{II Inhaltsverzeichnis}
\phantomsection
\addcontentsline{toc}{section}{\texorpdfstring{II \hspace{0.35em}Inhaltsverzeichnis}{Inhaltsverzeichnis}}
\addtocounter{section}{1}
\tableofcontents
\pagebreak


\rhead{VERZEICHNISSE}
\listoffigures
\pagebreak
\listoftables
\pagebreak
\renewcommand{\lstlistlistingname}{Listing-Verzeichnis}
{\labelsep2cm\lstlistoflistings}
\pagebreak

% ----------------------------------------------------------------------------------------------------------
% Abkürzungen
% ----------------------------------------------------------------------------------------------------------
%\section{Abkürzungsverzeichnis}
%\begin{acronym}[OSGi] % längste Abkürzung steht in eckigen Klammern
%	\setlength{\itemsep}{-\parsep} % geringerer Zeilenabstand
%	\acro{OSGi}{Open Service Gateway initiative}
%\end{acronym}
%\newpage

% ----------------------------------------------------------------------------------------------------------
% Inhalt
% ----------------------------------------------------------------------------------------------------------
% Abstände Überschrift
\titlespacing{\section}{0pt}{12pt plus 4pt minus 2pt}{-6pt plus 2pt minus 2pt}
\titlespacing{\subsection}{0pt}{12pt plus 4pt minus 2pt}{-6pt plus 2pt minus 2pt}
\titlespacing{\subsubsection}{0pt}{12pt plus 4pt minus 2pt}{-6pt plus 2pt minus 2pt}

% Kopfzeile
\renewcommand{\sectionmark}[1]{\markright{#1}}
\renewcommand{\subsectionmark}[1]{}
\renewcommand{\subsubsectionmark}[1]{}
\lhead{Kapitel \thesection}
\rhead{\rightmark}

\onehalfspacing
\renewcommand{\thesection}{\arabic{section}}
\renewcommand{\theHsection}{\arabic{section}}
\setcounter{section}{0}
\pagenumbering{arabic}
\setcounter{page}{1}

% ----------------------------------------------------------------------------------------------------------
% Einleitung
% ----------------------------------------------------------------------------------------------------------
\section{Kapitel: Herangehensweise}
\emph{Jakob}\\
Bei der Planung und der Herangehensweise an unser Projekt sind wir vermehrt auf Probleme gestoßen, welche in diesem Kapitel dargestellt werden.
...

\subsection{ursprüngliche Vorstellung und Konsequenzen}
\emph{Lars}\\
Unsere Unerfahrenheit in der Projektarbeit hat uns zu einem überstürzten und unüberlegten Projekteinstieg getrieben, bei dem wir uns in Themen eingearbeitet haben, ohne uns einen genauen Plan zu erstellen. Zu spät haben wir gemerkt, dass unsere bisherige Arbeit weder zielführend, noch in irgendeiner Weise sinnvoll war, weshalb wir uns zusammengesetzt und intensiv beraten haben. Aus dieser Beratung entstand zum ersten Mal ein und realistischer Plan bezüglich des Aufbaus unseres Programms. Auch die Aufteilung der verschiedenen Arbeitsschritte zwischen den  drei Gruppenmitgliedern fand hier spezifisch statt.

\subsubsection{Aufteilung}
\label{sec:Aufteilung}

\vspace{1em}
\begin{minipage}{\linewidth}
	\centering
	\includegraphics[width=0.7\linewidth]{Bilder/Projekt-Entwurf.jpg}
	\captionof{figure}[Erster realistischer Plan]{Erster realistischer Plan}
	\label{fig:plan1}
\end{minipage}
\vspace{1em}

In Abbildung \ref{fig:plan1} wird die Aufteilung der verschiedenen Aufgaben dargestellt. Der Client [\ref{sec:Client / GUI}], welcher von Lars programmiert werden sollte, umfasst allgemein gesagt dass GUI, also die Benutzeroberfläche, und dessen Kommunikation mit dem Server. Dieser wiederum soll von Robin eingerichtet und so gestaltet werden, dass er Anfragen kategorisiert und je nach Art der Anfrage das Datei-System beziehungsweise die Datenbank anspricht. Die soeben erwähnte Datenbank wird von Jakob erstellt und soll zunächst nur die Referenzen zu den Dateien im Datei-System abspeichern und verwalten.\\

\subsubsection{Client / GUI}
\label{sec:Client / GUI}

Die Tabelle \ref{tab:PlanungClient} ist die digitalisierte Version des Tafelbilds, welches bei der oben beschriebenen, intensiven Beratung entstanden ist. Grün steht dabei dafür, dass diese Kriterien automatisch erkannt werden sollen, rot markierte sollen erfragt werden und das blau markierte Kriterium ist eine optionale Angabe. Hierbei ist zu beachten, dass bei der Suche alle Angaben optional sind, sodass man, wenn man eine Suche ohne Kriterien startet, alle vorhandenen Dateien als Ergebnis ausgegeben bekommt.

\vspace{1em}
\begin{table}[!h]
	\centering
	\begin{tabular}{|p{3.3cm}|p{3.3cm}|p{3.3cm}|p{3.3cm}|}
		\hline
			\begin{center}
				\textbf{Ablage}
			\end{center}
			&
			\begin{center}
				\textbf{Suchen}
			\end{center}
			&
			\begin{center}
				\textbf{Löschen}
			\end{center}
			&
			\begin{center}
				\textbf{Ändern}
			\end{center}
		\\
		\hline
			\begin{itemize}[]
				\item[-]\textcolor{ForestGreen}{Typ}
				\item[-]\textcolor{ForestGreen}{Größe}
				\item[-]\textcolor{red}{Datei}
				\item[-]\textcolor{red}{Name}
				\item[-]\textcolor{red}{Datum}
				\item[-]\textcolor{blue}{zusätzliche Info}
			\end{itemize}
			&
			\begin{itemize}[]
				\item[]Kriterien:
				\item[-]\textcolor{red}{Name}
				\item[-]\textcolor{red}{Inhalt}
				\item[-]\textcolor{red}{Typ}
				\item[-]\textcolor{red}{Datum}
				\item[-]\textcolor{red}{Ort}
			\end{itemize}
			&
			\begin{itemize}[]
				\item[-]verbunden mit Suche
				\item[-]senden eines Löschbefehls
			\end{itemize}
			&
			\begin{itemize}[]
				\item[]Beinhaltet:
				\item[-]Suche
				\item[-]Ändern
				\item[-]Ablage
			\end{itemize}
			\\
		\hline
	\end{tabular}
	\caption{Planung des GUIs und dessen Funktionen}
	\label{tab:PlanungClient}
\end{table}

\subsubsection{Server}
\label{sec:Server}
Auch die Aufgaben des Servers wurden in unserem Tafelbild aufgeführt. Diese werden in Tabelle \ref{tab:PlanungServer} dargestellt.

\vspace{1em}
\begin{table}[!h]
	\centering
	Aufgaben des Servers in Verbindung mit...\\
	\begin{tabular}{|p{4cm}|p{4cm}|p{4cm}|}
		\hline
			\begin{center}
				\textbf{Client}
			\end{center}
			&
			\begin{center}
				\textbf{Datei-System}
			\end{center}
			&
			\begin{center}
				\textbf{Datenbank}
			\end{center}
		\\
		\hline
			\begin{itemize}[]
				\item[-]Auswerten der Metadaten bei der Ablage
				\item[-]Auswerten von Suchkriterien
				\item[-]Senden von Dateien bei der Suche
			\end{itemize}
			&
			\begin{itemize}[]
				\item[-]Ablage
				\item[-]Abruf
				\item[-]Löschen
			\end{itemize}
			&
			\begin{itemize}[]
				\item[-]Erstellen der Datenbank, falls keine vorhanden ist
				\item[-]Referenzen senden und erhalten
				\item[-]Einträge löschen
			\end{itemize}
		\\
		\hline
	\end{tabular}
	\caption{Planung des Servers}
	\label{tab:PlanungServer}
\end{table}

\subsubsection{Datenbank}
\label{sec:Datenbank}
Die Datenbank, so nahmen wir es uns vor, soll Dateien aufnehmen und den Abfragen entsprechend antworten. Beinhalten sollte sie zunächst nur den Dateinamen, sowie weitere Attribute, welche in Tabelle [\ref{tab:PlanungClient}] aufgeführt sind. Zusätzlich soll zu jeder Datei noch der jeweilige  Pfad gespeichert werden, sodass ein Zugriff auf die Datei gewährleistet werden kann.

\subsection{verspätete Planung}
\emph{Jakob}\\
...

\subsection{Zusammenarbeit und Kommunikation}
\emph{Jakob}\\
....

\pagebreak
% ----------------------------------------------------------------------------------------------------------
% Grundlagen
% ----------------------------------------------------------------------------------------------------------

\section{Kapitel: Grundlagen}
\emph{Jakob \emph{\&} Lars}\\
Im zweiten Kapitel dieser Dokumentation wird genauer auf die Themen eingegangen, in welche wir uns im Laufe des Projektkurses hineingearbeitet haben. Es wird nicht nur auf für das Projekt relevante Techniken, sondern auch auf solche, in die wir uns zwar eingearbeitet haben, sie jedoch nicht verwendet haben, eingegangen.

\subsection{\LaTeX{}}
\label{sec:LaTeX}
\emph{Robin}\\
...

\subsection{XAMPP}
\label{sec:XEMPP}
\emph{Robin}\\
...

\subsection{HTML}
\label{sec:HTML}
\emph{Lars}\\
HTML ist die Abkürzung für \glqq Hypertext Markup Language\grqq, was zu deutsch \glqq Hypertext-Auszeichnungssprache\grqq{} heißt. Es ist eine Programmiersprache, mit der man den Aufbau von Internetseiten bestimmt. Solche HTML-Dokumente stellen die Grundlage für das World Wide Web dar und werden von Browsern dargestellt.\cite{HTML}\cite{Hypertext_Markup_Language} Sie bestehen in der Regel aus drei Teilen.\cite{HTML/Dokumentstruktur_und_Aufbau} 

\vspace{1em}
\begin{lstlisting}[caption= Beispiel für ein einfaches HTML-Dokument, label=lst:HTML]
<html>
	<head>
		<meta charset="UTF-8">		
		<title> Titel</title>
	</head>
	<body>
		Sichtbarer Text auf der Webseite.
	</body>
</html>
\end{lstlisting}

Der erste Teil eines üblichen HTML-Dokuments ist die Dokumenttyp-Deklaration. In ihr werden Angaben zur verwendeten HTML-Version gegeben. Im \glqq head\grqq , welcher den zweiten Teil darstellt, werden  Kopfdaten, wie zum Beispiel der Titel der Seite oder andere, für den menschlichen Betrachter der Webseite zunächst nicht sichtbare, Informationen zur korrekten Darstellung des sichtbaren Teils der Webseite, angegeben.\cite{HTML/Kopfdaten} Anzuzeigende Inhalte werden in den \glqq body\grqq{}, den dritten Teil, geschrieben. Hier werden also sämtliche Texte, Verweise, Grafiken und so weiter eingefügt, die auf der Webseite sichtbar sein sollen.\cite{HTML/Dokumentstruktur_und_Aufbau} \\
In unserem Fall stellt HTML die Grundlage für die optische Gestaltung des GUIs dar.

\subsubsection{Frames}
\label{sec:Frames}
Eine hilfreiche Technik, die auch bei uns ihren Einsatz gefunden hat, heißt Frames. Diese Technik wurde 1996 von Netscape eingeführt. Mit ihr kann man mehrere Dateien gleichzeitig auf dem Bildschirm anzeigen lassen.\cite{HTML/Frames} Sie wurde jedoch im Oktober 2014 mit HTML5\cite{HTML5} aus dem Standard entfernt, da sie entscheidende Nachteile aufweist. Aufgrund des Verwendungszweckes unserer Webseite benutzen wir Frames, obwohl empfohlen wird, Server-seitig andere Techniken zum Auslagern von Teilen der Seite zu benutzen.\cite{HTML/Frames} Die stärksten Argumente waren, die simple Handhabung und die guten Gestaltungsmöglichkeiten mit dieser Technik. \\
Im Folgenden Beispiel wird ein sogenanntes Frameset dargestellt, bei dem der Bildschirm, mit dem Attribut \glqq rows\grqq{} in zwei Zeilen aufgeteilt wird, wobei die obere 20\% der Pixel einnimmt und die untere den Rest, also 80\%. Alternativ kann man den Bildschirm auch in Spalten aufteilen, dies geschieht mit dem Attribut \glqq cols\grqq . Des Weiteren wird mit dem Attribut \glqq border\grqq{} die Breite des Randes zwischen den jeweiligen Frames angegeben.

\vspace{1em}
\begin{lstlisting}[caption= Beispiel für Frames in HTML, label=lst:HTML]
<html>
	<head>
		<title>Titel</title>
	</head>
	<frameset rows="20%,*" border="1">
		<frame src="Quelle1.html">
		<frame src="Quelle2.html">
	</frameset>
</html>
\end{lstlisting}

\subsubsection{Tabellen}
\label{sec:Tabellen}
Tabellen in HTML [\ref{sec:HTML}] bieten gute, einfache und vielseitige Möglichkeiten, Internetseiten zu strukturieren. Sie wurden im Januar 1997 mit HTML 3.2 ins Standardrepertoire von HTML aufgenommen.\cite{Hypertext_Markup_Language} \\
Das Folgende Beispiel beinhaltet eine Tabelle mit zwei Zeilen und zwei Spalten. Tabellen in HTML werden Zeile für Zeile definiert. Eine Zeile beginnt mit $<$tr$>$ und wird mit $<$/tr$>$ beendet. Mithilfe der Befehle $<$td$>$ und $<$/td$>$ werden die Tabelleneinträge, also die Spalten in den jeweiligen Zeilen, definiert.

\vspace{1em}
\begin{lstlisting}[caption= Beispiel für Tabellen in HTML, label=lst:HTML]
<html>
	<head>
		<meta charset="UTF-8">		
		<title> Titel</title>
	</head>
	<body>
		<table>
			<tr>
				<td>
					Oben links
				</td>
				<td>
					Oben rechts
				</td>
			</tr>
			<tr>
				<td>
					Unten links
				</td>
				<td>
					Unten rechts
				</td>
			</tr>
		</table>
	</body>
</html>
\end{lstlisting}

\subsubsection{Formulare}
\label{sec:Formulare}
Formulare sind ein Element von HTML, [\ref{sec:HTML}] das es ermöglicht Daten zu erfassen und über das Hypertext Transfer Protocol per XMLHttpRequest, HTTP-GET oder HTTP-POST zur Verarbeitung an einen Server zu senden.\cite{Webformular} In unserem Programm kam letzteres zum Einsatz. Man kann in HTML zwar Formulare definieren und erstellen, für eine Verarbeitung und Auswertung der Eingaben ist jedoch eine andere Programmiersprache, wie zum Beispiel Javascript [\ref{sec:JavaScript}] oder PHP [\ref{sec:PHP}] nötig.\cite{HTML/Formulare/Form} \\
In unserem Fall haben wir Formulare in Form von Anmeldeformularen, Suchfiltern und auch als Möglichkeit zum Hochladen von Dateien eingesetzt.\\
In dem folgenden Beispiel ist die Implementation eines Formulars in HTML dargestellt. Zu sehen ist ein Formular, welches ein Label, also den Text \glqq Suchgbegriff\grqq{} enthält. Die Zugehörigkeit des darauf folgenden Eingabefelds zum Label wird durch das Attribut \glqq name\grqq{} festgelegt. Das zweite \glqq input\grqq{} Statement erstellt den Bestätigungsknopf, den man drücken muss, um das Formular abzusenden. Beim Absenden des Formulars wird die Datei oder die Internetseite aufgerufen, die im Kopf des Formulars unter dem Attribut \glqq action\grqq{} steht.

\vspace{1em}
\begin{lstlisting}[caption= Beispiel für Formulare in HTML, label=lst:HTML]
<html>
	<head>
		<title>
			Titel
		</title>
	</head>
	<body>
		<form action="action.php">
 			 <label for="begriff">Suchbegriff</label>
	 		 <input type="text" name="begriff">
	 		 
	 		 <input type="submit" name="Submit" value="Suchen">
		</form>
	</body>
</html>
\end{lstlisting}

\subsubsection{JavaScript}
\label{sec:JavaScript}
Bei JavaScript handelt es sich um eine interpretierende Programmier- beziehungsweise Skriptsprache, die 1995 von Netscape entwickelt wurde.\cite{JavaScript}\cite{wiki/JavaScript} JavaScript ist sehr verbreitet, da sich in allen modernen Browsern Interpreter für die Sprache befinden. Es wird hauptsächlich Client-seitig verwendet und ermöglicht es, dynamischen Einfluss auf Webseiten zu nehmen.\cite{JavaScript-JS} Für unser Programm kam die Sprache besonders häufig aufgrund des, durch sie ermöglichten, einfachen Umgangs mit Variablen und Funktionen zum Einsatz.

\subsection{PHP}
\label{sec:PHP}
\emph{Robin}\\
...

\subsection{Apache}
\label{sec:Apache}
\emph{Lars}\\
...

\subsection{Editoren und integrierte Entwicklungsumgebungen}
\label{sec:Editoren}
\emph{Lars}\\
Editoren werden zum Schreiben von Texten, wie zum Quellcodes, benutzt. gute Editoren helfen das Programmieren zu vereinfachen, indem sie gewisse Schlüsselwörter, sowie Befehle, farblich hervorheben, eine Autovervollständigung, eine Such- und Ersetzfunktion und indem sie den Quellcode automatisch einrücken, sowie eine Schnittstelle für Plugins darstellen.\cite{Texteditor} \\
Neben Editoren gibt es auch integrierte Entwicklungsumgebungen. Diese bestehen aus einer Sammlung an Computerprogrammen, mit denen es möglich ist Software ohne die Verwendung vieler einzelner Programme zu entwickeln. Durch sie werden nicht nur Tippfehler verhindert, sondern auch Arbeitsschritte und somit Zeit bei der Softwareentwicklung gespart.\cite{Integrierte_Entwicklungsumgebung}\cite{Medienbruch}
Bei der Entwicklung unseres Programms kamen die Editoren und integrierten Entwicklungsumgebungen Eclipse [\ref{sec:Eclipse}] (genau genommen Eclipse Neon IDE), Notepad++ [\ref{sec:Notepad++}] und Texmaker [\ref{sec:Texmaker}] zum Einsatz.

\subsubsection{Eclipse}
\label{sec:Eclipse}
\emph{Robin}\\
...

\subsubsection{Notepad ++}
\label{sec:Notepad++}
\emph{Robin}\\
...

\subsubsection{Texmaker}
\label{sec:Texmaker}
\emph{Robin}\\
...

\subsection{GitHub}
\label{sec:GitHub}
\emph{Jakob}\\
Durch die Arbeit mit GitHub, wird das gemeinsame, nicht zwingend parallele, Arbeiten möglich. GitHub verwaltet die Quellcodes, welche die Benutzer hochladen so, dass jeder, der über Zugriff auf das Projekt verfügt, dieses weiter führen kann. -Lars

\subsection{Dropzone}
\label{sec:Dropzone}
\emph{Lars}\\
Ursprünglich beschreibt der Begriff \glqq Dropzone\grqq{} einen geheimen Speicherort für maschinell gestohlene Daten, wie zum Beispiel Passwörter und Kontodaten.\cite{wiki/Dropzone} \\
In unserem Fall ist die Dropzone jedoch ein Script, welches in JavaScript [\ref{sec:JavaScript}] geschrieben wurde. Dieses dient zur einfachen Gestaltung des Datei-Uploads mittels HTML. Es soll die optische Anpassung des Eingabefelds vereinfachen. 
Aufgrund ihrer Komplexität, welche sich leider erst nach und nach herauskristallisierte, haben wir uns schlussendlich gegen die Dropzone und für einen herkömmlichen Datei-Upload mittels HTML-Formular [\ref{sec:Formulare}] entschieden.

\subsection{Java-Servlet}
\label{sec:Java-Servlet}
\emph{Jakob}\\
Im Folgenden werde ich erläutern, wie man ein Servlet in Eclipse über einen Apache Tomcat Server erstellen kann.

\subsubsection{Einrichtung in Eclipse}
Vorausgesetzt für eine erfolgreiche Einrichtung ist die bereits erfolgte Installation von Eclipse Neo in der Java EE Version und Apache Tomcat v 9.0. 
Nachdem  diese erfolgt ist, ist es nun möglich in Eclipse einen neuen Server einzurichten, wobei man unter dem Ordner Apache die installierte Tomcat Version vorfindet, welche hier  als Servertyp verwendet wird.
Nachdem dies geschehen ist, richten sie ein Dynamic Web Project ein und innerhalb desselben erstellen sie im Java Resources/src  Ordner ein neues Servlet, indem sie nun programmieren können.

Wenn nun aber die Fehlermeldung  „cannot resolve“ im Bezug auf die vorgegebenen import-Zeilen auftaucht müssen sie außerdem noch die servlet-api.jar downloaden und diese dann in den Eigenschaften ihres Dynamic Web Projects unter Java Build Path in der Kategorie Libraries mit der Funktion „Add External JARs“ hinzufügen. 

...

\subsubsection{Vorteile}
...

\subsubsection{Nachteile}
...

\subsection{JSON}
\label{sec:JSON}
\emph{Jakob}\\
Ein weiteres Thema, in welches ich mich eingearbeitet habe war die Erstellung und das Auslesen von Dateien
des Formates JSON mit Eclipse.
...

\subsubsection{Erstellen}
...

\subsubsection{Auslesen}
...

\subsubsection{Vorteile}
...

\subsubsection{Nachteile}
...

\subsection{MYSQL}
\label{sec:MYSQL}
\emph{Jakob}\\
Nachdem obige Thematiken angesichts unseres erneuerten Projektplans irrelevant geworden waren, habe ich mich 
mit MYSQL beschäftigt.
...

\subsubsection{Vorteile}
...

\subsubsection{Datenbank-Entwurf}
Im Folgenden möchte ich auf das Entwerfen einer MYSQL-Datenbank eingehen.
...

\subsubsection{Verwaltung einer Datenbank mit PHP}
Um eine Datenbank in unser Projekt zweckmäßig einzurichten, musste ich mich in eine MYSQL-Einbindung mittels PHP einlesen, welche ich nun darstellen möchte.
...

\pagebreak

% ----------------------------------------------------------------------------------------------------------
% Kapitel 3 - Projekt
% ----------------------------------------------------------------------------------------------------------
\section{Projekt}
\label{sec:Projekt}
\emph{Lars}\\
In diesem Kapitel wird nun unser Projekt dargestellt, indem zunächst das Programm im Gesamten beschrieben wird. Daraufhin wird die Arbeit von Lars am GUI, die Arbeit von Robin am Interface und Datei-System und zuletzt die Arbeit von Jakob am Interface und an der Datenbank dargestellt. 
...

\subsection{Datei-Verwaltungs-Programm}
\emph{Lars}\\
Darstellung Ergebnis...

\subsection{GUI}
\emph{Lars}\\
Darstellung GUI...

\subsection{Engine I: Interface und Datei-System}
\emph{Robin}\\
Darstellung Interface und File-System...

\subsection{Engine II: Interface und Datenbank}
\emph{Jakob}\\
Darstellung Interface und Datenbank...
\pagebreak
% ----------------------------------------------------------------------------------------------------------
% Kapitel
% ----------------------------------------------------------------------------------------------------------
\section{Fazit}
\label{sec:Fazit}
Im Folgenden wird jeder von uns ein Fazit geben, indem er darstellt, was er aus dem Projekt gelernt hat, was ihm gefallen hat, Sachen die er gut findet und so weiter.

\subsection{}
\emph{Lars}\\
...

\subsection{}
\emph{Robin}\\
...

\subsection{}
\emph{Jakob}\\
...

\pagebreak
% ----------------------------------------------------------------------------------------------------------
% Literatur
% ----------------------------------------------------------------------------------------------------------

%\addcontentsline{tocnumbered}{section}{Literaturverzeichnis}

\begin{thebibliography}{999}

%\bibitem[Anzeigename]{Stichwort, dass man mit \cite aufruft}
%\url{link zur Quelle}, Zugriff: 16.06.2017

\bibitem{HTML}
\url{https://wiki.selfhtml.org/wiki/HTML}, \\
Zugriff: 15.06.2017

\bibitem{Hypertext_Markup_Language}
\url{https://de.wikipedia.org/wiki/Hypertext\_Markup\_Language},\\
Zugriff: 15.06.2017

\bibitem{HTML/Dokumentstruktur_und_Aufbau}
\url{https://wiki.selfhtml.org/wiki/HTML/Dokumentstruktur\_und\_Aufbau}, \\
Zugriff: 15.06.2017

\bibitem{HTML/Kopfdaten}
\url{https://wiki.selfhtml.org/wiki/HTML/Kopfdaten}, \\
Zugriff: 15.06.2017

\bibitem{HTML/Frames}
\url{https://wiki.selfhtml.org/wiki/HTML/Frames}, \\
Zugriff: 15.06.2017

\bibitem{HTML5}
\url{https://de.wikipedia.org/wiki/HTML5}, \\
Zugriff: 15.06.2017

\bibitem{Webformular}
\url{https://de.wikipedia.org/wiki/Webformular}, \\
Zugriff: 15.06.2017

\bibitem{JavaScript}
\url{http://www.searchenterprisesoftware.de/definition/JavaScript}, \\
Zugriff: 15.06.2017

\bibitem{wiki/JavaScript}
\url{https://de.wikipedia.org/wiki/JavaScript}, \\
Zugriff: 15.06.2017

\bibitem{JavaScript-JS}
\url{http://www.itwissen.info/JavaScript-JavaScript-JS.html}, \\
Zugriff: 15.06.2017

\bibitem{HTML/Formulare/Form}
\url{https://wiki.selfhtml.org/wiki/HTML/Formulare/Form}, \\
Zugriff: 15.06.2017

\bibitem{Integrierte_Entwicklungsumgebung}
\url{https://de.wikipedia.org/wiki/Integrierte_Entwicklungsumgebung}, \\
Zugriff: 16.06.2017

\bibitem{Texteditor}
\url{https://de.wikipedia.org/wiki/Texteditor}, \\
Zugriff: 16.06.2017

\bibitem{Medienbruch}
\url{https://de.wikipedia.org/wiki/Medienbruch}, \\
Zugriff: 16.06.2017

\bibitem{wiki/Dropzone}
\url{https://de.wikipedia.org/wiki/Dropzone}, \\
Zugriff: 16.06.2017



%\bibitem{}
%\url{}, \\
%Zugriff: 16.06.2017


\end{thebibliography}
%\section{Quellenverzeichnis}
%\printbibliography
%\renewcommand\refname{Quellenverzeichnis}
%\bibliographystyle{alpha}
%\bibliography{bibo}
\pagebreak


% ----------------------------------------------------------------------------------------------------------
% Anhang
% ----------------------------------------------------------------------------------------------------------
\pagenumbering{Roman}
\setcounter{page}{1}

\lhead{Anhang \thesection}
\begin{appendix}

\section*{Anhang}
\phantomsection
\addcontentsline{toc}{section}{Anhang}
\addtocontents{toc}{\vspace{-0.5em}}


\end{appendix}


\newpage
\thispagestyle{empty}
\begin{center}
	\vspace*{5em}
	\huge\textbf{Erklärung}\\
\end{center}
\vspace{2em}
Hiermit versichere ich, dass ich meine Dokumentation selbständig verfasst und keine anderen als die angegebenen Quellen und Hilfsmittel benutzt habe.

\vspace{4em}
\begin{minipage}{\linewidth}
	\begin{tabular}{p{15em}p{15em}}
		Datum: &  .......................................................\\
		& \centering (Unterschrift)\\
	\end{tabular}
\end{minipage}

\end{document}

%Die Quellen befinden sich in der Datei \textit{bibo.bib}. Ein Buch- und eine Online-Quelle sind beispielhaft eingefügt. [Vgl. \cite{buch}, \cite{online}]
%Abkürzungen lassen sich natürlich auch nutzen (\ac{OSGi}). Weiter oben im Latex-Code findet sich das %Verzeichnis.


%Beispiele
%
%
%----------Listings----------
%
%
% \ref{lst:arduino}.
%
%\vspace{1em}
%\begin{lstlisting}[caption=Arduino Beispielprogramm, label=lst:arduino]
%int ledPin = 13;
%void setup() {
%    pinMode(ledPin, OUTPUT);
%}
%void loop() {
%    digitalWrite(ledPin, HIGH);
%    delay(500);
%    digitalWrite(ledPin, LOW);
%    delay(500);
%}
%\end{lstlisting}
%
%
%
%----------Aufzählung----------
%
%
%\begin{compactitem}
%	\item Nur
%	\item ein
%	\item Beispiel.
%\end{compactitem}
%
%
%
%----------Tabelle----------
%
%
%\vspace{1em}
%\begin{table}[!h]
%	\centering
%	\begin{tabular}{|l|l|l|}
%		\hline
%		\textbf{Name} & \textbf{Name} & \textbf{Name}\\
%		\hline
%		1 & 2 & 3\\
%		\hline
%		4 & 5 & 6\\
%		\hline
%		7 & 8 & 9\\
%		\hline
%	\end{tabular}
%	\caption{Beispieltabelle}
%	\label{tab:beispiel}
%\end{table}
%
%
%
%----------Grafik/Abbildung----------
%
%
%\vspace{1em}
%\begin{minipage}{\linewidth}
%	\centering
%	\includegraphics[width=0.7\linewidth]{Bilder/layering-osgi.png}
%	\captionof{figure}[OSGi Architektur]{OSGi Architektur\footnotemark }
%	\label{fig:osgi}
%\end{minipage}
%\footnotetext{Quelle: \url{http://www.osgi.org/Technology/WhatIsOSGi}}



%section*{}  -> Section, die nicht im Inhaltsverzeichnis aufgeführt wird



